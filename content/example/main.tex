\chapter{Example Chapter}
\blindtext[3]

\section{First subsection}
\blindtext[4]
\begin{equation}
	\begin{aligned}
		\int \limits_\Omega \bs{\varepsilon}(\delta\bs{u}) : \bs{C} : \bs{\varepsilon}(\bs{u})\intd{\Omega}
		&= \int \limits_\Omega \delta\bs{u} \bs{p}\intd{\Omega}
		 + \int \limits_{\Gamma_N} \delta\bs{u} \bs{q}\intd{\partial\Omega}
		 + \int \limits_{\Gamma_D} \delta\bs{u} \bs{q}\intd{\partial\Omega} \\
		\land \quad \bs{u} &= \overline{\bs{u}} \quad \forall \bs{x} \in \Gamma_D \\
	\end{aligned}
\end{equation}
\blindtext[2]
\begin{figure}
	\begin{center}
		\begin{tikzpicture}

	% b.c.
	\draw[black, decorate, decoration = {
		markings,
		mark = between positions 0cm and 0.56cm step 0.8mm with {
			\draw[black] (0, 0) -- (0.1, -0.15);
		}
		pre = curveto,
		post = curveto,
		transform = {shift only},
	}]
	(1.5, 0.5) to[out = 10, in = -90]
	++(1.5, 2) to[out = 90, in = 0]
	++(-1, 1) to[out = 180, in = 30]
	++(-1, -1) to[out = -150, in = 90]
	++(-1, -1) to[out = -90, in = 190]
	cycle;

	\draw[black, decorate, decoration = {
		markings,
		mark = between positions 4.5cm and 5cm step 1.2mm with {
			\draw[black, -stealth] (0, 0) -- (-0.2, 0.2);
		}
		pre = curveto,
		post = curveto,
		transform = {shift only},
	}]
	(1.5, 0.5) to[out = 10, in = -90]
	++(1.5, 2) to[out = 90, in = 0]
	++(-1, 1) to[out = 180, in = 30]
	++(-1, -1) to[out = -150, in = 90]
	++(-1, -1) to[out = -90, in = 190]
	cycle;

	% org. problem
	\draw[black, fill = black!20!white]
	(1.5, 0.5) to[out = 10, in = -90]
	++(1.5, 2) to[out = 90, in = 0]
	++(-1, 1) to[out = 180, in = 30]
	++(-1, -1) to[out = -150, in = 90]
	++(-1, -1) to[out = -90, in = 190]
	cycle;

	% fikt. domain
	\draw[black, fill = black!20!white] (3.75, 0.25) rectangle ++(3.5, 3.5);

	\draw[black, fill = white]
	(5.5, 0.5) to[out = 10, in = -90]
	++(1.5, 2) to[out = 90, in = 0]
	++(-1, 1) to[out = 180, in = 30]
	++(-1, -1) to[out = -150, in = 90]
	++(-1, -1) to[out = -90, in = 190]
	cycle;

	% combind domain
	\draw[black, fill = black!20!white] (8.25, 0.25) rectangle ++(3.5, 3.5);

	\draw[black, densely dashed]
	(10, 0.5) to[out = 10, in = -90]
	++(1.5, 2) to[out = 90, in = 0]
	++(-1, 1) to[out = 180, in = 30]
	++(-1, -1) to[out = -150, in = 90]
	++(-1, -1) to[out = -90, in = 190]
	cycle;

	% mesh
	\begin{scope}[shift = {(4.5cm, 0)}]

		\draw[black, densely dashed, fill = black!20!white]
		(10, 0.5) to[out = 10, in = -90]
		++(1.5, 2) to[out = 90, in = 0]
		++(-1, 1) to[out = 180, in = 30]
		++(-1, -1) to[out = -150, in = 90]
		++(-1, -1) to[out = -90, in = 190]
		cycle;


		\draw[black] (8.25, 0.25) -- ++(2.8, 0) -- ++(0, 0.7) -- ++(0.7, 0) -- ++(0, 2.8) -- ++(-2.1, 0) -- ++(0, -0.7) -- ++(-0.7, 0) -- ++(0, -0.7) -- ++(-0.7, 0) -- cycle;

		% mesh
		\draw[black] (8.25, 0.95) -- ++(2.8, 0);
		\draw[black] (8.25, 1.65) -- ++(3.5, 0);
		\draw[black] (8.95, 2.35) -- ++(2.8, 0);
		\draw[black] (9.65, 3.05) -- ++(2.1, 0);

		\draw[black] (8.95, 0.25) -- ++(0, 2.1);
		\draw[black] (9.65, 0.25) -- ++(0, 2.8);
		\draw[black] (10.35, 0.25) -- ++(0, 3.5);
		\draw[black] (11.05, 0.95) -- ++(0, 2.8);

		\foreach \x in {8.25, 8.95, 9.65, 10.35, 11.05}
			{
				\foreach \y in {0.25, 0.95, 1.65, 2.35}
					{
						\draw[black, fill = white] (\x, \y) circle (0.08);
					}
					\draw[black, fill = white] ({\x + 0.7}, 3.05) circle (0.08);
			}

		\foreach \x in {9.65, 10.35, 11.05, 11.75}
			{
				\draw[black, fill = white] (\x, 3.75) circle (0.08);
			}

		\foreach \y in {0.95, 1.65, 2.35}
			{
				\draw[black, fill = white] (11.75, \y) circle (0.08);
			}


		\begin{scope}[shift = {(8.5, 0)}]
			\draw[black, decorate, decoration = {
				markings,
				mark = between positions 0cm and 0.56cm step 0.8mm with {
					\draw[black] (0, 0) -- (0.1, -0.15);
				}
				pre = curveto,
				post = curveto,
				transform = {shift only},
			}]
			(1.5, 0.5) to[out = 10, in = -90]
			++(1.5, 2) to[out = 90, in = 0]
			++(-1, 1) to[out = 180, in = 30]
			++(-1, -1) to[out = -150, in = 90]
			++(-1, -1) to[out = -90, in = 190]
			cycle;

			\draw[black, decorate, decoration = {
				markings,
				mark = between positions 4.5cm and 5cm step 1.2mm with {
					\draw[black, -stealth] (0, 0) -- (-0.2, 0.2);
				}
				pre = curveto,
				post = curveto,
				transform = {shift only},
			}]
			(1.5, 0.5) to[out = 10, in = -90]
			++(1.5, 2) to[out = 90, in = 0]
			++(-1, 1) to[out = 180, in = 30]
			++(-1, -1) to[out = -150, in = 90]
			++(-1, -1) to[out = -90, in = 190]
			cycle;
		\end{scope}
	\end{scope}

	% cup and equal sign
	\node[] at (3.4, 2) {$\boldsymbol{\bigcup}$};
	\node[] at (7.75, 2) {$\boldsymbol{=}$};
	\node[] at (12.25, 2) {$\boldsymbol{\to}$};

	% domains
	\node[] at (1.5, 1.5) {$\Omega_\text{phy}$};
	\node[below right] at (3.8, 3.7) {$\Omega_\text{fict}$};
	\node[above right] at (8.3, 0.3) {$\Omega_\cup$};

	\node[] at (10, 1.5) {$\alpha = 1$};
	\node[below right] at (8.3, 3.7) {$\alpha \approx 0$};

	% b.c.
	\node[] at (1.2, 3.7) {$\boldsymbol{t}_0$};
	\node[] at (1.05, 3.05) {$\Gamma_N$};

	\node[] at (2.1, 0.2) {$\boldsymbol{u}_0$};
	\node[] at (2.6, 0.7) {$\Gamma_D$};

\end{tikzpicture}

	\end{center}
	\caption{This is an example image. The visualization has been done using the \texttt{TikZ} package.}
	\label{fig:example_tikz_image}
\end{figure}
\subsection{First subsection}
\Cref{fig:example_tikz_image} \blindtext[3]
\subsubsection{First subsubsection}
\blindtext[3]
\subsubsection{Second subsubsection}
\blindtext[3]
\subsection{Second subsection}
\blindtext[4]
\begin{figure}
	\begin{center}
		\begin{tikzpicture}[spy using outlines={black, line width = 1, densely dashed, thick, rectangle, size=2cm, magnification=2, connect spies}]

	\begin{axis} [
		font={\footnotesize},
		axis lines = box,
		xlabel = $x$,
		ylabel = $y$,
		width = 0.75\textwidth,
		height = 8cm,
		max space between ticks = 40,
		grid = major,
		grid style = {densely dashed, line width = 0.1pt},
		minor x tick num = 9,
		minor y tick num = 9,
		cycle list name = plotColorListMark,
		legend pos = north east,
		legend cell align={left},
		domain = -10:10,
		smooth,
	]

		\addplot { -1.0 * atan( 1.0 * x ) };
		\addplot { -1.1 * atan( 1.1 * x ) };
		\addplot { -1.2 * atan( 1.2 * x ) };
		\addplot { -1.3 * atan( 1.3 * x ) };
		\addplot { -1.4 * atan( 1.4 * x ) };
		\addplot { -1.5 * atan( 1.5 * x ) };
		\addplot { -1.6 * atan( 1.6 * x ) };
		\addplot { -1.7 * atan( 1.7 * x ) };
		\addplot { -1.8 * atan( 1.8 * x ) };

		\legend{
			$\alpha = 1.0$,
			$\alpha = 1.1$,
			$\alpha = 1.3$,
			$\alpha = 1.4$,
			$\alpha = 1.5$,
			$\alpha = 1.6$,
			$\alpha = 1.7$,
			$\alpha = 1.8$,
		};

	\end{axis}

	\spy[height = 2.5cm, width = 2.5cm] on (2.5, 5.2) in node[fill = white] at (1.5, 1.5);

\end{tikzpicture}

	\end{center}
	\caption{This is an example plot. The visualization has been done using the \texttt{pgfplots} package.}
	\label{fig:example_tikz_image}
\end{figure}
\cite{bathe2007finite} \blindtext[5]

\begin{notebox}{Example of 'notebox'}
	\blindtext
\end{notebox}

\begin{importantbox}{Example of 'importantbox'}
	\blindtext
\end{importantbox}

\begin{alertbox}{Example of 'alertbox'}
	\blindtext
\end{alertbox}

\begin{lstlisting}[caption = {Example of 'lstlisting' with Julia code}, label = {code:example}, float]
function unit_bounding_box( dimension::Int64 )
	return BoundingBox(
		dimension, zeros( dimension ), 2.0 .* ones( dimension )
	)
end # function
\end{lstlisting}

\todo{Example of a 'TODO'. These can be placed during writing to add comments on open tasks.}
